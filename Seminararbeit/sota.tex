\chapter{State of the Art}
In der Geschichte der Autoindustrie hat sich im Bereich Fahrer-Cockpit einiges getan und auch die HCI hat �ber die letzten Jahrzehnte User Interfaces im Automobil erforscht. Durch die wachsende Anzahl an Interaktionsm�glichkeiten f�r den Fahrer steigt leider auch die Gefahr, dass dem Fahrer durch die Ablenkung Fehler passieren.\\

Im groben k�nnen die Interaktionsbereiche des Fahrers unterteilt werden in die Windschutzscheibe (z.B. Navi), das Amaturenbrett, die Mittelkonsole, das Lenkrad, dem Boden (z.B. Pedale) und der Peripherie (z.B. Seitenspiegel).\cite{kern} Diese Seminararbeit besch�ftigt sich von nun an im wesentlichen mit der Interaktionsschnittstelle Lenkrad.

\section{Interaktionsm�glichkeiten am Lenkrad}
Die Hauptfunktionalit�t eines Lenkrads ist das Bestimmen der Richtung, in welche sich das Fahrzeug bewegen soll. Daf�r  wird das Lenkrad in die jeweilige Richtung gedreht. Es finden sich jedoch weitere Interaktionsm�glichkeiten am Lenkrad wieder, welche verschiedene Aufgaben erf�llen.\\

So steuert man nicht nur die Richtung, sondern auch andere Funktionalit�ten des Fahrzeuges am Lenkrad an. Man kann beispielsweise die Hupe, Blinker, Lichtanlage, Scheibenwischer, Radio, Sprachsteuerung, Lautsprecheinrichtung, Lautst�rke, Tempomat und weitere Funktionen bet�tigen.\\

Daf�r wurden auch verschiedene Modalit�ten erforscht und eingesetzt, wie diese enorme Anzahl an Aufgaben am effektivsten zu bedienen ist. Das wohl g�ngigste Beispiel ist die Hupe, welche bet�tigt wird indem man in die Mitte des Lenkrads dr�ckt. Weitere Modalit�ten sind zum Beispiel die Hebel links und rechts hinter dem Lenkrad, welche sich oft nach unten, oben, vor und zur�ck dr�cken und im Uhrzeigersinn und gegen den Uhrzeigersinn drehen lassen. Damit werden meist die Lichtanlage, Blinker und Scheibenwischer des Autos bedient. Au�erdem befinden sich heute auch oft noch Kn�pfe auf den Hebeln, um weitere Einstellungen vorzunehmen.

Doch nicht nur Hebel und Druckkn�pfe sondern auch Einstellr�der, Knebelkn�pfed, Drehr�der, etc. finden am Lenkrad eine Einsatzm�glichkeiten.

\section{Related Work}
Die HCI besch�ftigt sich seit langem, wie die gro�e Anzahl an Funktionalit�ten am Lenkrad am effizientesten in Hinsicht Sicherheit, Effektivit�t und Benutzerfreundlichkeit zu bew�ltigen sind. So wurde beispielsweise auch die R�ckseite des Lenkrades in Betracht gezogen und verschiedenen M�glichkeiten vorgestellt diesen Raum zu nutzen.\cite{murer}\\

Ein weiterer Ansatz besch�ftigt sich mit Touch Interfaces und erforscht den Design Space, wie daumengesteuerte Touch Interfaces am besten bedient werden k�nnen.\cite{werner, gonzalez, pfeiffer} Auch mit der Interaktion mittels Sprache und Gesten hat man sich bereits besch�ftigt.\cite{pfleging, doring}

\chapter{Einleitung}
Die HCI kann man in viele Aufgabengebiete unterteilen und es gibt auch diverse Definitionen daf�r. Auf jeden Fall beinhaltet die HCI jedoch das Studieren und Entwickeln von verbesserten Arten der User Experience und neuer Interaktionsparadigmen, sowie die Anwendung von interdisziplin�ren Methoden f�r benutzerzentriertes Design. Auch in der Lehrveranstaltung HCI Studio wurde versucht, diese Ziele umzusetzen, indem wir verschiedenste Interaktionskontexte untersuchten, mit dem Ziel diese Kontexte f�r eine optimale Experience zu verbessern und benutzerzentrierte Methoden f�r innovative und optimierte Systeme anzuwenden.\\

Das Thema der Lehrveranstaltung war die Exploration des Design Spaces "`Lenkrad im Auto"' und bestand aus Einzel- und Gruppenaufgaben zu obigem Thema. Ziel war es, durch benutzerzentriertes Design einen innovativen und optimierten Prototypen zu implementieren, welcher den Design Space des Lenkrads im Auto verwendet und das Benutzererlebnis erh�ht.\\

Daf�r klassifizierten wir derzeitige Interaktionsm�glichkeiten am Lenkrad und recherchierten aktuelle Forschungen alternativer Interaktionsm�glichkeiten am Lenkrad. Mit Hilfe der Lehrveranstaltungsleiter, welche uns mit verschiedenen Methoden der Ideenfindung unterst�tzten, konnten wir aus diesen Ergebnissen verschiedene Konzepte erarbeiten, wie man am Lenkrad interagieren k�nnte.\\

In der Gruppe wurde dann eine Interaktionstechnologie fokusiert und jeder setzte das Design und Prototyping von 3 innovative Visualisierungen um. Im Anschluss wurde der Prototyp in Form einer kleinen Nutzerstudie evaluiert.

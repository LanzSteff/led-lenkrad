\chapter{Evaluierung}
Unsere Implementierungen wurden daraufhin anhand einer Nutzerstudie getestet. Daf�r lie�en wir die anderen Lehrveranstaltungsteilnehmer und Mitarbeiter der HCI \& Usability Unit mit dem Fahrsimulator gewisse Strecken fahren und baten sie danach einen kurzen Fragebogen auszuf�llen, um herauszufinden, ob Nutzer diese Technologie privat verwenden w�rden und ob ihnen die Technologie beim Fahren geholfen hat. Dabei wurden folgende sieben Fragen gestellt.
\begin{enumerate}
	\item Ich h�tte das System gerne in meinem Auto
	\item Wenn ich das System in meinem Auto habe, werde ich es benutzen
	\item Die Nutzung des Systems wird meine Fahrleistung verbessern
	\item Ich finde das System ist beim Fahren unpraktisch
	\item Die Nutzung des Systems verbessert meine Bedienung des Autos
	\item Ich finde das System ist beim Fahren n�tzlich
	\item Meine Interaktion mit dem System ist klar verst�ndlich
\end{enumerate}
Diese Fragen konnten mit "`Ich stimme dem 1. Sehr 2. Ziemlich 3. Mittel 4. Wenig 5. nicht zu"' beantwortet werden.\\

Alle drei Visualisierungen wurden hier �hnlich beantwortet und es konnte evaluiert werden, dass die Testuser den Prototyp gerne im Auto h�tten und ihn auch verwenden w�rden. Es war f�r die Tester leicht mit dem System zu interagieren und das System wurde als n�tzlich empfunden. Jedoch konnte der Prototyp keine Verbesserung der Fahrleistung bewirken und auch die Bedienung des Autos wird dadurch nicht vereinfacht. Keine der Testpersonen empfand die Visualisierung als unpraktisch.

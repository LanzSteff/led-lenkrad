\chapter{Konzepte}
Im Rahmen der Lehrveranstaltung wurden verschiedene Konzepte erarbeitet. Daf�r wurde das Lenkrad zu beginn kritisch betrachtet und es kamen Ideen auf, wie anstatt eines Lenkrads das Fahrzeug mit einem Joystick zu lenken. Oder die Eigenschaften von Lenkrad und Pedalen wurden vertauscht und das Auto wurde mit zwei Pedalen (links und rechts) gelenkt und mit dem Lenkrad - von sich weg dr�ckend (schneller) und zu sich her ziehend (langsamer) - beschleunigt.

Eine weitere Idee war eine Steuerung �hnlich einem Panzer, indem man mit einem Hebel links die beiden linken R�der beschleunigte bzw. verlangsamte und analog einem Hebel rechts hatte.

Einer Idee nach konnte man mit einem Touchpad an der Vorderseite in der Mitte des Lenkrads mittels Symbolen die Blinker, den Radio und weitere Elemente im Auto bedienen.\\

Um f�r weitere Konzepte und Ideen offener zu werden, sollten wir das Lenkrad unter dem Thema Tiere auch exzentrisch betrachten. Dadurch entstanden Erfindungen, ein Auto mit verschiedenen Tierger�uschen zu steuern. So war das Br�llen eines Tigers gleichzusetzen mit dem Bet�tigen des Gaspedals und des Meckern eines Schafes entsprach dem Bremsen. Weiters wurde in einer Idee das Fahrer-Cockpit komplett mit einem gro�en Hamsterrad ersetzte, mit welchem das Fahrzeug gesteuert wurde.\\

Bei der Ideenfindung f�r unseren Prototypen haben wir versucht verschiedene Interfaces miteinzubeziehen. Indem man beispielsweise Funktionen mittels Schiebereglern an der R�ckseite des Lenkrads bet�tigte oder an der Vorderseite ein Touch-Interface befestigt wird, womit man einige Aufgaben durchf�hren kann.

Durchgesetzt hat sich die Idee, Anzeigen im Armaturenbrett und in der Mittelkonsole mittels eines LED-Streifens am �u�eren des Lenkrads zu visualisieren.
